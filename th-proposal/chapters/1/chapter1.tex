\chapter{บทนำ}
\section{ที่มาและความสำคัญ}
    \centering{\bf{\textit{(ตัวอย่างปัญหา...)}}} \\
    \thaijustify{
        ทักษะการเขียนโปรแกรมคอมพิวเตอร์เป็นทักษะที่สำคัญสำหรับวิชาชีพในสาขาวิทยาการคอมพิวเตอร์ เป็นเหตุให้นักศึกษาจำนวนมากหันมาสมัครหรือลงวิชาเขียนโปรแกรมคอมพิวเตอร์เพิ่มมากขึ้น ทำให้การเรียนการสอนวิชาเขียนโปรแกรมคอมพิวเตอร์มีประสิทธิภาพที่ต่ำลงเพราะจำนวนผู้เรียนมากขึ้น อาจารย์ผู้สอนไม่สามารถจะดูแลนักศึกษาได้ครบทุกคน การตรวจงานและให้คะแนนใช้ระยะเวลานาน เพราะอาจารย์ผู้สอนจะต้องนำไฟล์งานของนักศึกษามาตรวจทีละไฟล์ อีกทั้งยังมีโอกาสเกิดข้อผิดพลาดในการตรวจอีกด้วย
    }
    \thaijustify{
        จากปัญหาดังกล่าว...
    }
    \thaijustify{
        โดยซอฟต์แวร์จะ...
    }
\section{วัตถุประสงค์}
    ทางกลุ่มเรา ได้ดำเนินโครงการพัฒนาซอฟต์เเวร์ดังกล่าว ด้วยวัตถุประสงค์ดังต่อไปนี้
    \begin{enumerate}
        \item เพื่อ...
        \item เพื่อ...
        \item เพื่อ...
    \end{enumerate}

\section{ขอบเขตของโครงงาน}
    \begin{enumerate}
        \item ซอฟต์แวร์รองรับการแสดงผลเป็นภาษาอังกฤษและภาษาไทย...
        \item ซอฟต์แวร์เน้นการใช้งานบนหน้าจอมือถือ...
        \item มีระบบ...
        \item ระบบรองรับ
        \begin{enumerate}
            \item เงื่อนไข...
            \item สามารถ...
        \end{enumerate}
    \end{enumerate}

%: NOTE: This should not be included in thesis, but in proposals
\section{ประโยชน์ที่คาดว่าจะได้่รับ}
    หลังจากสิ้นสุดโครงการ ทางคณะผู้จัดทำคาดหวังว่า โครงการดังกล่าวจะให้ผลประโยชน์ต่อทั้งตนเเละผู้อื่น ดังต่อไปนี้
    \begin{enumerate}
        \item ทางคณะผู้จัดทำจะได้รับ...
        \item อาจารย์ได้...
        \item ทางภาควิชาได้....
    \end{enumerate}

\pagebreak
\section{ตารางการดำเนินงาน}
    \centering{\bf{\textit{(ตัวอย่าง Gantt Chart...)}}} \\
    \thaijustify{
        ในส่วนนี้เป็นแผนและตารางเวลาของแต่ละขั้นตอน...
    }
\begin{table}[!h]
    \centering
    \caption{ตารางการดำเนินการในภาคการศึกษาที่ 1}
    \label{tbl:gantt1}
    \begin{ganttchart}[
        x unit = 0.5cm,
        y unit chart = 1.2cm,
        y unit title = 0.6cm,
        title height = 1,
        vgrid={*{3}{black, dotted}, *1{black, dashed}},
        hgrid={*1{black, dashed}},
        bar top shift = 0.1, 
        %bar height = 0.8,
        bar label node/.append style={
            align=right,
            text width=width("7. จัดทำรายงานของภาคการศึกษาที่ 1")
        }
    ]{1}{20}
        \gantttitle{สิงหาคม}{4} \gantttitle{กันยายน}{4} \gantttitle{ตุลาคม}{4} \gantttitle{พฤศจิกายน}{4} \gantttitle{ธันวาคม}{4} \\
        \gantttitlelist{1,...,4}{1} \gantttitlelist{1,...,4}{1} \gantttitlelist{1,...,4}{1} \gantttitlelist{1,...,4}{1} \gantttitlelist{1,...,4}{1} \\
        \ganttbar{ศึกษาค้นคว้า วิเคราะห์หาปัญหาและที่มา}{2}{2} \\
        \ganttbar{เสนอหัวข้อโครงงาน}{3}{3} \\
        \ganttbar{ศึกษาค้นคว้า หาข้อมูลที่เกี่ยวข้อง}{3}{4} \\
        \ganttbar{นำเสนอโครงการให้กับอาจารย์ที่ปรึกษา}{4}{4} \\
        \ganttbar{จัดทำข้อเสนอโครงการ}{5}{9} \\
        \ganttbar{นำเสนอข้อเสนอโครงการ}{11}{11} \\
    \end{ganttchart}
\end{table}

\begin{table}[!h]
    \centering
    \caption{ตารางการดำเนินการในภาคการศึกษาที่ 2}\label{tbl:gantt2}
    \begin{ganttchart}[
       x unit = 0.5cm,
        y unit chart = 1.2cm,
        y unit title = 0.6cm,
        title height = 1,
        vgrid={*{3}{black, dotted}, *1{black, dashed}},
        hgrid={*1{black, dashed}},
        bar top shift = 0.1, 
        %bar height = 0.8,
        bar label node/.append style={
            align=right,
            text width=width("7. จัดทำรายงานของภาคการศึกษาที่ 1")}
]   {1}{20}
        \gantttitle{มกราคม}{4} \gantttitle{กุมภาพันธ์}{4} \gantttitle{มีนาคม}{4} \gantttitle{เมษายน}{4} \gantttitle{พฤษภาคม}{4} \\
        \gantttitlelist{1,...,4}{1} \gantttitlelist{1,...,4}{1} \gantttitlelist{1,...,4}{1} \gantttitlelist{1,...,4}{1} \gantttitlelist{1,...,4}{1} \\
        \ganttbar{พัฒนาซอฟต์แวร์และระบบ (ต่อจากภาคการศึกษาที่ 1)}{1}{1} \\
    \end{ganttchart}
\end{table}
\pagebreak