
%%%%%%%%%%%%%%%%%%%%%%%% TEMPLATE INFO %%%%%%%%%%%%%%%%%%%%%%%%%%%%%%
%: Template Name = cpe-thesis-en
%: Version Name  = th-vectier-1.1
%: Credits
%: - Peerapon Siripongwutikorn, CPE, 2016
%: - Wuttipat Chokananatasab, FIBO, 2016
%: - Thanin Srithai, CPE, 2021
%: - Jatetanan Kanchanawat, CPE, 2023
%:
%: This is a LaTeX template used in CPE, KMUTT thesis.
%%%%%%%%%%%%%%%%%%%%%%%%%%%%%%%%%%%%%%%%%%%%%%%%%%%%%%%%%%%%%%%%%%%%%
%: Instructions:
%: Run at the command line:
%: - xelatex <filename> or latexmk -xelatex <filename> to compile
%: tex files
%: - bibtex <filename> or biber <filename> to compile bib file
%: Note: Run a few times to generate the output pdf file
%:
%%%%%%%%%%%%%%%%%%%%%%%%%% REPORT START %%%%%%%%%%%%%%%%%%%%%%%%%%%%%
\documentclass[12pt,one side,openright,a4paper]{cpe-thesis-en}

%%%%%%%%%%%%%%%%%%%%%% Packages and Configs %%%%%%%%%%%%%%%%%%%%%%%%%

\usepackage{color}
\usepackage{pgfgantt}                   % Gantt chart
\usepackage{xcolor}
\usepackage{polyglossia}                % Thai language script and format
\usepackage{caption}                    % Figures and tables captions

\usepackage{etoolbox}                   % Patching commands for macros creating i.e. preto
\usepackage{longtable}                  % Long continuous table
\usepackage{ragged2e}                   % Forcing figure
\usepackage{float}                      % ER and checkmark
\usepackage{tikz}                       % Drawing graphs and symbols
\usepackage{subfig}                     % Subfigures
\usepackage{keyval}                     % Subfigures alignment
\usepackage[export]{adjustbox}          % Subfigures aLignment
\usepackage{cleveref}                   % Clever label references

\usepackage{lmodern}
\usepackage{listings}                   % Listings as code blocks
\usepackage{packages/listings-go}       % Listings support for Golang
\usepackage{packages/listings-js}       % Listings support for JavaScript and ES6
\usepackage{packages/listings-yaml}     % Listing support for YAML

\usepackage{multirow}

%: # Biblatex
%: Bib manager; instead of using \bibliography{}
%: NOTE: Biblatex defines typesetting onto compilation cache.
%: So recompile from scratch (Delete all aux, bbl, and other output files, then recompile)
%: everytime you stopped using biblatex

% \usepackage[
%     backend=bibtex,     % Biber or Bibtex
%     style=ieee,         % Bib standard
%     sorting=none,       % Sort by appearance in paper
%     % sorting=ynt,       % Sort by year, name...
%     % sortcites=true,    % Some other example options
%     dateabbrev=true,
%     urldate=long,
%     block=none,
%     indexing=false,
%     citereset=none,
%     isbn=true,
%     url=true,
%     language=english,
%     doi=true,           % Prints DOI
%     natbib=true         % if you need Natbib functions
% ]{biblatex}

%%%%%%%%%%%%%%%%%%% Custom macros & Settings %%%%%%%%%%%%%%%%%%%%%%%%
%: From this part to \begin{document}
%: Do not modify unless you know what you are doing...


%: # Biblatex macro
%: Uncomment all of these if you're using biblatex

%: Setup URL access date string
% \DefineBibliographyStrings{english}{%
%     urlseen = {accessed}
% }

%: Reference bib files
%: Make sure you reference correct bib file...
% \addbibresource{bib/codern.bib}

%: # Typography
%: Define 1st and 2nd language
\setdefaultlanguage{english}
\setotherlanguage{thai}
\emergencystretch=10pt

%: Define fonts for code blocks
\newfontfamily\codefont[Scale=0.95]{CourierPrime.ttf}[
    Path=fonts/,
    Extension=.ttf,
    BoldFont=*-Bold,
    ItalicFont=*-Italic,
    BoldItalicFont=*-BoldItalic,
]

\newcommand{\engjustify}[1]{%
  \par\hspace{30pt}\justifying
  #1
}
\newcommand{\thaijustify}[1]{%
  \par\hspace{30pt}\justifying
  #1
}

%: # Tikz Settings
%: Tikz is used for drawing simple edge and node graph
%: Import shapes for usage
\usetikzlibrary{er, positioning, shapes.geometric, arrows, matrix}

%: Define checkmark symbol
\def\checkmark{\tikz\fill[scale=0.4](0,.35) -- (.25,0) -- (1,.7) -- (.25,.15) -- cycle;}

%: # Listings Settings
%: Listings are used for displaying the program or
%: codes in reports as statement

%: Define text color for code blocks
\definecolor{dkgreen}{rgb}{0,0.6,0}
\definecolor{gray}{rgb}{0.5,0.5,0.5}
\definecolor{mauve}{rgb}{0.58,0,0.82}
\definecolor{AllColor}{HTML}{F4A300}
\definecolor{ShayathipColor}{HTML}{5B9BD5}
\definecolor{RaminColor}{HTML}{70AD47}

%: Go code snippets setting
\lstset{
    frame=tb,
    language=Golang,
    aboveskip=3mm,
    belowskip=3mm,
    showstringspaces=false,
    columns=flexible,
    basicstyle={\small\codefont},
    numbers=left,
    numberstyle=\small\color{gray},
    keywordstyle=\color{blue},
    commentstyle=\color{dkgreen},
    stringstyle=\color{mauve},
    breaklines=false,
    breakatwhitespace=false,
    tabsize=3
}

%: # Math macro
%: Fraction settings
\renewcommand{\topfraction}{0.85}
\renewcommand{\textfraction}{0.1}

%: # Define theorem and proof
\newtheorem{theorem}{Theorem}
\newtheorem{lemma}{Lemma}
\newtheorem{corollary}{Corollary}

\def\QED{\mbox{\rule[0pt]{1.5ex}{1.5ex}}}
\def\proof{\noindent\hspace{2em}{\itshape Proof: }}
\def\endproof{\hspace*{\fill}~\QED\par\endtrivlist\unskip}

%\newenvironment{proof}{{\sc Proof:}}{~\hfill \blacksquare}
%% The hyperref package redefines the \appendix. This one
%% is from the dissertation.cls
%\def\appendix#1{\iffirstappendix \appendixcover \firstappendixfalse \fi \chapter{#1}}
%\renewcommand{\arraystretch}{0.8}

%%%%%%%%%%%%%%%%%%%%%% Report Definitions %%%%%%%%%%%%%%%%%%%%%%%%%%%
%: Customize below to suit your needs
%: The optional ones can be left blank.

%: # Project Type
%: Enter 'Project' / 'Independent Study' / 'Thesis'
\def\worktype{Project}
\def\thaiworktype{ปริญญานิพนธ์}

%: # Credits
%: Enter number based on your subject's credits
\def\disscredit{18}

%: # Fulfillment; 'Degree' or 'Subject'
%: Set to 'Degree' for senior project or thesis
%: 'Subject' for any subject project report i.e. soft-eng final report
\def\fulfillment{Degree}

%: # Titles
\def\disstitleone{Query Assistance}
\def\thaidisstitleone{Query Assistance}
\def\disstitletwo{}
\def\disstitlethree{}
\def\thaidisstitletwo{}
\def\thaidisstitlethree{}

%: # Authors
\def\dissauthor{Ms. Shayathip Dumkum}
\def\thaidissauthor{นางสาวชยาทิพย์ ดำคำ}
\def\dissauthortwo{Mr. Ramin Suchatnitikul}
\def\thaidissauthortwo{นายรามิล สุชาตินิติกุล}
\def\dissauthorthree{}
\def\thaidissauthorthree{}

%: # Diploma
%: If you're still an undergraduate having not
%: graduated from any degree; leave these fields empty
%: Example: B.Eng. (Computer Engineering)
\def\dissdiplomaone{}
\def\dissdiplomatwo{}
\def\dissdiplomathree{}
\def\thaidissdiplomaone{}
\def\thaidissdiplomatwo{}
\def\thaidissdiplomathree{}

%: Advisors
\def\dissadvisor{Asst.Prof. Santitham Prom-on, Ph.D.} % Advisor
\def\thaidissadvisor{ผศ.ดร.สันติธรรม พรหมอ่อน}

%: Leave empty if you have no co-advisor
\def\disscoadvisor{} % Co-advisor (optional)
\def\disscoadvisortwo{} % Co-advisor (optional)
\def\thaidisscoadvisor{}
\def\thaidisscoadvisortwo{}

%: # Committees
%: Note that senior projects have no committee chair
\def\disscommitteechair{} % Committee chair (optional)
\def\thaidisscommitteechair{}

%: Leave the following empty if no person is in that position
%: Your project or independent study's committee
%: Example of appropriate advisor/committee name entries
%: Example 1: Asst. Prof. Dr.Ing Priyakorn Pusawiro
%: Example 2: Jaturon Harnsomburana, Ph.D.
\def\disscommitteetwo{Asst.Prof. Rajchawit Sarochawikasit, Ph.D} % Committee member
\def\disscommitteethree{Unchalisa Taetragool, Ph.D.} % Committee member (optional)
\def\disscommitteefour{Asst. Prof. Jumpol Polvichai, Ph.D.} % Committee member (optional)
\def\thaidisscommitteetwo{ผศ.ราชวิชช์ สโรชวิกสิต}
\def\thaidisscommitteethree{ดร.อัญชลิสา แต้ตระกูล}
\def\thaidisscommitteefour{ผศ.ดร. จุมพล พลวิชัย}

%: # Degree
%: Degree that you're pursuing
\def\dissdegree{Bachelor of Engineering} % Name of the degree
\def\thaidissdegree{วิศวกรรมศาสตรบัณฑิต}
\def\dissdegreeabrev{B.Eng.} % Abbreviation of the degree
\def\dissyear{2024}  % Year of submission
\def\thaidissyear{2567} % Year of submission (in B.E.)

%: # Department and Institution Information
\def\institute{King Mongkut's University of Technology Thonburi}
\def\fieldofstudy{Computer Engineering}
\def\department{Computer Engineering}
\def\faculty{Faculty of Engineering}
\def\thaiinstitute{มหาวิทยาลัยเทคโนโลยีพระจอมเกล้าธนบุรี}
\def\thaifieldofstudy{วิศวกรรมคอมพิวเตอร์}
\def\thaidepartment{วิศวกรรมคอมพิวเตอร์}
\def\thaifaculty{วิศวกรรมศาสตร์}

%%%%%%%%%%%%%%%%%%% Front Page / Signature Page %%%%%%%%%%%%%%%%%%%%%
\begin{document}
\pdfstringdefDisableCommands{
  \let\MakeUppercase\relax
}

\genfrontpages{kmutt.jpg}{2.8cm}
%%%%%%%%%%%%%%%%%%%%%%%%% English abstract %%%%%%%%%%%%%%%%%%%%%%%%%%

\def\abstcontent{
  In the fast-paced business world, data plays a critical role in driving decision-making, solving problems, and improving operations. However, relying on specialists to retrieve data can be both costly and time-consuming. To address this challenge, Query Assistance is proposed as a tool to simplify data access, enabling users to efficiently retrieve data from databases without requiring SQL expertise. Designed for fast-growing travel technology companies like Agoda, Query Assistance aims to save time, cost, and resources while improving data accessibility for non-technical staff.

  Query Assistance leverages Large Language Models (LLMs) and LangChain as its core technologies. LLMs are utilized to debug SQL queries, optimize query performance, and translate natural language inputs into SQL queries. LangChain provides a robust framework for building agents that understand user inputs, generate responses, and interact with data. By integrating metadata includes table names, table descriptions, columns name, columns types, etc. from OpenMetadata, the tool ensures accurate column and table selection, enabling informed decision-making. This innovative tool is designed to enhance SQL operations, streamline data retrieval, and empower users to access data quickly and effectively, driving efficiency and productivity in data-driven organizations.

}
\def\abstkeyword{
  SQL query language, Large Language Model (LLM), LangChain, GPT, Query optimization, Multi-agent

}

%%%%%%%%%%%%%%%%%%%%%%%%%%% Thai abstract %%%%%%%%%%%%%%%%%%%%%%%%%%%

\def\thabstcontent{
  \justify{
    ในโลกธุรกิจที่มีการเปลี่ยนแปลงอย่างรวดเร็ว ข้อมูลถือเป็นองค์ประกอบสำคัญที่ช่วยสนับสนุนการตัดสินใจ การแก้ปัญหา และการปรับปรุงประสิทธิภาพการดำเนินงาน อย่างไรก็ตาม การพึ่งพาผู้เชี่ยวชาญในการดึงข้อมูลจากฐานข้อมูลนั้นเป็นกระบวนการที่ใช้ทั้งเวลาและค่าใช้จ่ายสูง เพื่อแก้ไขปัญหาดังกล่าว โครงการนี้จึงได้พัฒนาเครื่องมือที่ชื่อว่า “Query Assistance” ซึ่งมีวัตถุประสงค์เพื่ออำนวยความสะดวกในการเข้าถึงข้อมูล ช่วยให้ผู้ใช้งานสามารถดึงข้อมูลจากฐานข้อมูลได้อย่างมีประสิทธิภาพ โดยไม่จำเป็นต้องมีความรู้ด้านภาษา SQL

    Query Assistance อาศัยเทคโนโลยี Large Language Models (LLMs) และเฟรมเวิร์ก LangChain เป็นแกนหลัก โดย LLMs ถูกนำมาใช้ในการแก้ไขข้อผิดพลาดของคำสั่ง SQL ปรับปรุงประสิทธิภาพในการประมวลผล และแปลงข้อความภาษาธรรมชาติให้เป็นคำสั่ง SQL ในขณะที่ LangChain ช่วยในการจัดการตัว Agent ที่สามารถเข้าใจคำสั่งของผู้ใช้ สร้างคำตอบ และโต้ตอบกับข้อมูลได้อย่างมีประสิทธิภาพ นอกจากนี้ ระบบยังได้เชื่อมต่อกับ OpenMetadata เพื่อนำข้อมูลเมตาของตารางและคอลัมน์มาใช้ในการประมวลผล ซึ่งช่วยเพิ่มความแม่นยำในการเลือกข้อมูลและการตัดสินใจ เครื่องมือนี้ถูกออกแบบมาเพื่อช่วยให้องค์กรที่ขับเคลื่อนด้วยข้อมูล โดยเฉพาะบริษัทเทคโนโลยีด้านการท่องเที่ยวที่มีการเติบโตอย่างรวดเร็ว เช่น Agoda สามารถลดต้นทุน ประหยัดเวลา และเพิ่มประสิทธิภาพในการเข้าถึงข้อมูลให้แก่บุคลากรที่ไม่ได้มีพื้นฐานด้านเทคนิค

  }
}
\def\thabstkeyword{
  SQL query language, Large Language Model (LLM), LangChain, GPT, Query optimization, Multi-agent
}

\genabstract

%%%%%%%%%%%%%%%%%%%%%%%%% Acknowledgments %%%%%%%%%%%%%%%%%%%%%%%%%%%

\def\prefacecontent{
  We would like to express our deepest gratitude to our professor, Asst.Prof. Santitham Prom-on, Ph.D., and the committee for their invaluable guidance, support, and encouragement throughout this project. Their mentorship and constructive feedback have been instrumental in shaping the direction and success of our work. We are truly grateful for their time, effort, and dedication in helping us improve and refine this project.

  We would also like to extend my heartfelt thanks to Agoda for providing us with the incredible opportunity to work on this project. The resources, tools, and collaborative environment offered by Agoda have been essential in enabling us to learn, grow, and contribute meaningfully. Lastly, We are deeply appreciative of the advice and help we received from our mentors and colleagues. Their insights, encouragement, and expertise have been invaluable in overcoming challenges and enhancing the quality of this project. This experience has been both enriching and rewarding, and We are sincerely thankful to everyone who supported us along the way.

}

\genpreface

%%%%%%%%%% Table of contents and report elements lists %%%%%%%%%%%%%%
%: The commands below automatically generate the table
%: of content, list of tables, list of figures, and list of listings
%: # Table of content
\tableofcontents

%: # Table of report elements (auto-generated)
\newcommand{\tableraglf}[1]{\RaggedRight{#1}}
\listoftables
\listoffigures
% \listofprograms % consists of 'lst' from package listings

%: # List of symbols (manually added)
% \listofsymbols
% \justifying{
%   \centering{\bf{\textit{(Example list of symbols...)}}} \\
%   %: You can adjust width to suit your word length
%   \begin{tabular}{@{}p{0.07\textwidth}p{0.7\textwidth}p{0.1\textwidth}}
%     \textbf{SYMBOLS} &                                     & \textbf{UNIT} \\[0.2cm]
%     $\alpha$         & \tableraglf{Test variable}\hfill    & m$^2$         \\
%     $\lambda$        & \tableraglf{Interarival rate}\hfill & jobs/second   \\
%     $\mu$            & \tableraglf{Service rate}\hfill     & jobs/second   \\
%   \end{tabular}
% }

%: # List of vocab and technical terms (manually added)
\listofvocab
\begin{flushleft}
  \centering{\bf{\textit{(Example list of vocabs...)}}} \\
  %: You can adjust width to suit your word length
  \begin{tabular}{@{}p{1.2in}@{\hspace{0.08in}=\extracolsep{0.2in}}p{4in}}
    LLM   & Large Language Model \\
    SQL & Structured Query Language \\
    API  & Application programming interface \\
    REST & Representational State Transfer \\
    MVC & Model-View-Controller \\
    MAS & Multi-Agent System \\
    RAG & Retrieval-Augmented Generation \\
    AI & Artificial Intelligence \\
    ORM & Object-Relational Mapping \\
    CaaS & Container as a Service \\
    CI & Continuous Integration \\
    ML & Machine Learning \\
    UI & User Interface \\
    ORDBMS & Object-Relational Database Management System \\
    HDFS & Hadoop Distributed File System \\
    MPP & Massively Parallel Processing \\
    CBO & Cost-Based Optimizer \\
    OLAP & Online Analytical Processing \\
    CPU & Central Processing Unit \\
    RDD & Resilient Distributed Dataset \\
    UML & Unified Modeling Language \\
    CSV & Comma Separated Values \\
  \end{tabular}
\end{flushleft}

%%%%%%%%%%%%%%%%%%%%%%%%%%%% Contents %%%%%%%%%%%%%%%%%%%%%%%%%%%%%%%
%: Insert your contents chapter by chapter here
%: using \input{path\to\chapter\tex}

\chapter{Introduction}


% \emph{}

\section{Background} 

Undeniably, data plays a critical role in driving companies especially for swiftly moving business worlds. The data could direct the company by offering insights that can solve problems, lead to improvements, help making an informed decision making, understand the customer preference, enhance marketing strategies, cost reduction, risk management, and so many other things. Relying on specialists to retrieve data efficiently is costly and time-consuming. It would therefore be rather advantageous and efficient to create a query support tool that assists users in retrieving data from the database, particularly for fast-growing travel technology companies like Agoda.
The potential benefits of Query Assistance are that it would help in simplifying data access by enabling employees to quickly retrieve data without requiring an understanding of SQL. The user can retrieve data quickly and efficiently using the Query Assistance in which helps to save cost, time, and resources from querying the data.

\section{Proposed Method}
    \subsection{Approach}
    Query Assistance is the tool for creating the SQL query based on the user input. The Query Assistance is capable of debugging  the SQL query, optimizing  the SQL query, and translating the natural language to the SQL query. Initially, the Query Assistance will be built as an API platform, allowing other platforms to call it for execution. It will then be integrated into Superset, enabling users to utilize it when they encounter issues in SQL Lab. The Query Assistance works by using Large Language Models (LLMs) and LangChain as a core function. Large Language Models (LLMs) play a crucial role in this project by improving and expediting SQL operations. The LLMs will be specifically used to fix errors in SQL code, ranging from syntax problems to incorrect table or column references, and automatically generate SQL queries based on natural language input. They will also be used to modify existing SQL statements to improve accuracy or optimize performance Complementing the capabilities of LLMs, LangChain provides a robust framework for building applications where these models can function as an agent. The agent performs tasks such as understanding user inputs, generating responses, and interacting with data. LangChain also offers a library for creating custom tools, aiming to optimize the use of agents and tools to achieve project objectives. The agents extend their capabilities beyond mere text generation by interacting with tools that enable them to fetch real-time data, perform computations, and execute specific commands. To further enhance data interaction, LangChain utilizes metadata stored in OpenMetadata to analyze and select the appropriate columns and tables. This integration ensures that the agents can make informed decisions and effectively manage data interactions, thereby achieving the project's goals efficiently.
    \subsection{Objectives}
        \begin{itemize}
        \item  Simplify data access: Empower workers to retrieve data easily without needing to understand SQL.
        \item  Enhance query accuracy: Automatically correct SQL statements, from syntax to selecting the correct columns and tables.
        \item  Boost efficiency: Save time by quickly generating flawless SQL queries, optimizing both productivity and cost.
        \item  Facilitate advanced queries: Allow users to effortlessly create complex queries and explore advanced SQL techniques with minimal errors.
        \end{itemize}

\section{Scope of Work}
This project will be divided into three phases. The first phase focuses on error handling to ensure that we ultimately provide the correct query for the user. This includes correcting all syntax errors, accommodating syntax differences across databases, and fixing column names, table names, and typos in the query. The second phase focuses on optimizing the query such as using the partition column, decreasing memory consumption, and reducing the query execution time. The third and final phase aims to translate natural language into SQL query language.

\section{Original Engineering Content}
The original engineering content covered in the project includs the following: Software Engineering, Data Analysis, AI, and User Interface 
    \begin{itemize}
        \item  Software
        \begin{itemize}
            \item API Design and Implementation
        \end{itemize}
        \item  Data Analysis
        \begin{itemize}
            \item Performance Metrics
        \end{itemize}
        \item  System Integration
        \begin{itemize}
            \item Superset Integration
            \item Slack Integration
        \end{itemize}
    \end{itemize}

\pagebreak
\section{Project Schedule}
    \begin{table}[!h]
        \centering % Ensures the table is centered
        \caption{Project schedule in the first semester}
        \label{tbl:gantt1}
        \resizebox{\textwidth}{!}{ % Resizes the Gantt chart to fit within the page width
            \begin{ganttchart}[
                x unit = 0.5cm,
                y unit chart = 1.2cm,
                y unit title = 0.6cm,
                title height = 1,
                vgrid={*{3}{black, dotted}, *1{black, dashed}},
                hgrid={*1{black, dashed}},
                bar top shift = 0.1, 
                bar height = 0.8,
                bar label font=\small, % Adjust font size for labels
                bar label node/.append style={
                    align=right,
                    text width=6cm % Set a fixed width for text wrapping
                }
            ]{1}{20}
                \gantttitle{Aug}{4} \gantttitle{Sep}{4} \gantttitle{Oct}{4} \gantttitle{Nov}{4} \gantttitle{Dec}{4} \\
                \gantttitlelist{1,...,4}{1} \gantttitlelist{1,...,4}{1} \gantttitlelist{1,...,4}{1} \gantttitlelist{1,...,4}{1} \gantttitlelist{1,...,4}{1} \\
                \ganttbar{Project Discussion with company}{1}{1} \\
                \ganttbar{Project Idea}{2}{2} \\
                \ganttbar{Project Brainstorm}{2}{3} \\
                \ganttbar{Design architecture for project}{3}{4} \\
                \ganttbar{Set up project}{4}{4} \\
                \ganttbar{Design sequence diagram for create API to add and update Table details}{5}{6} \\
                \ganttbar{Implement query assist table details API}{4}{4} \\
                \ganttbar{Design sequence diagram for tools in Query Assist}{5}{6} \\
                \ganttbar{Implement tools for Query Assist}{7}{7} \\
                \ganttbar{Merge and Debug}{7}{8} \\
                \ganttbar{Prompt Tuning \& Monitor result}{8}{9} \\
                \ganttbar{CI/CD + Deploy}{9}{9} \\
                \ganttbar{Slack Integration}{10}{15} \\
                \ganttbar{Slack Tracking Performance + Enhance the MVP}{12}{13} \\
                \ganttbar{Merge PostgreSQL}{14}{15} \\
                \ganttbar{Merge Query Assist v1 with v2}{16}{17} \\
                \ganttbar{POC and Design the Optimization function}{15}{15} \\
                \ganttbar{Implement Optimization function}{18}{20}  % No extra \\
            \end{ganttchart}
        }
    \end{table}
\pagebreak
    \begin{table}[!h]
        \centering
        \caption{Project schedule in the second semester}
        \label{tbl:gantt2}
        \begin{ganttchart}[
            x unit = 0.5cm,
            y unit chart = 1.2cm,
            y unit title = 0.6cm,
            title height = 1,
            vgrid={*{3}{black, dotted}, *1{black, dashed}},
            hgrid={*1{black, dashed}},
            bar top shift = 0.1, 
            %bar height = 0.8,
            bar label node/.append style={
                align=right,
                % text width=width("Present findings to advisors")
            }
        ]{1}{20}
            \gantttitle{Jan}{4} \gantttitle{Feb}{4} \gantttitle{Mar}{4} \gantttitle{Apr}{4} \gantttitle{May}{4} \\
            \gantttitlelist{1,...,4}{1} \gantttitlelist{1,...,4}{1} \gantttitlelist{1,...,4}{1} \gantttitlelist{1,...,4}{1} \gantttitlelist{1,...,4}{1} \\
            \ganttbar{Implement Optimization function}{1}{2} \\
            \ganttbar{Prompt Tuning \& Monitor result}{3}{5} \\
            \ganttbar{Superset Integration}{4}{5} \\
            \ganttbar{Get feedback + Enhance the functions}{5}{5} \\
            \ganttbar{POC and Design the Phase 3 functions}{6}{7} \\
            \ganttbar{Implement Phase 3 functions}{8}{14} \\
            \ganttbar{Prompt Tuning \& Monitor result}{14}{15} \\
            \ganttbar{Get feedback + Enhance the functions}{15}{17} \\
            \ganttbar{Implement Multi Agent}{16}{18} \\
            \ganttbar{Debug}{18}{19} \\
            \ganttbar{Improve and Finalize}{19}{20}  % No extra \\
        \end{ganttchart}
    \end{table}
\section{Deliverables for Term 1}
\begin{itemize}
    \item  Overall system design
    \begin{itemize}
        \item Architecture Design
        \item Use Case Diagram
        \item Sequence Diagram
        \item Flow Chart
    \end{itemize}

    \item  Query Assist Phase 1
    \begin{itemize}
        \item Correcting all syntax errors
        \item Accommodating syntax differences across databases
        \item Fixing typos in column names, and table names
    \end{itemize}

    \item  Integrate into Slack
    \item Deploy MVC (Minimum viable product)
\end{itemize}
\section{Deliverables for Term 2}
\begin{itemize}
    \item  Performance Metrics

    \item  Query Assist Phase 2
    \begin{itemize}
        \item Auto add partition column
        \item Decreasing memory consumption
        \item Reducing the query execution time
    \end{itemize}
    \item Query Assist Phase 3
    \begin{itemize}
        \item Translate natural language to SQL query
    \end{itemize}
\end{itemize}
\pagebreak
\chapter{Background Theory and Related Work}

THIS IS AN EXAMPLE. ALL SECTIONS BELOW ARE OPTIONAL. PLEASE CONSULT YOU ADVISOR AND DESIGN YOUR OWN SECTION

\textthai{หัวข้อต่าง ๆ ในแต่ละบทเป็นเพียงตัวอย่างเท่านั้น หัวข้อที่จะใส่ในแต่ละบทขึ้นอยู่กับโปรเจคของนักศึกษาและอาจารย์ที่ปรึกษา}

This is how you add the website URL: \url{http://www.cpe.kmutt.ac.th}

Explain theory, algorithms, protocols, or existing research works and tools related to your work.

You can cite your references like this: \cite{booch87}, or multiplie cite like this: \cite{meyer2000, atwoodmd}

\section{Recommender Systems}

    \begin{table}[!h]
    \caption{test table method1}\label{tbl:method1}
        \begin{tabular}{c|c|l|rr} \hline\hline
            Center & Center & left aligned & Right & Right aligned \\ \hline\hline
            Center & Center & left aligned & Right & Right aligned \\ \hline
            Center & Center & left aligned & Right & Right aligned \\ 
            Center & Center & left aligned & Right & Right aligned \\ \hline
            Center & Center & left aligned & Right & Right aligned \\ \hline\hline
        \end{tabular}
    \end{table}

    You can place any report elements and refer to it like Figure~\ref{tbl:method1}, \ref{fig:oop-concept}
    The figure and table numbering will be run and updated automatically when you add/remove tables/figures from the document.

    \begin{figure}[H]
        \centering
        \includegraphics[width=7cm]{chapters/2/figures/react-lifecycle.png}
        \caption[Aspects of OOPs]{Aspects of OOPs from~\cite{apollo22oop}}
        \label{fig:oop-concept}
    \end{figure}
    
    \subsection{Algorithm I}
    Add more subsections as you want.
    \subsection{Algorithm II}
        Add more subsections as you want.
        \subsubsection{Step I}
            You can use subsection too!
        \subsubsection{Step II}
            This is the farthest level of subsection we permitted. (We support only 4th level)


\pagebreak
\chapter{Proposed Work}
\section{Introduction}
This chapter outlines the design and methodologies undertaken for the project. The chapter provides detailed explanations beginning with the design phase, including use cases, UML diagrams, and flowcharts.This is followed by a detailed, step-by-step account of each procedure. The chapter concludes with the methods used to evaluate the performance of the Query Assistance.

\section{Project Functionality}
    \subsection{System requirements}
    \begin{itemize}
        \item  Fix SQL query
        \begin{itemize}
            \item Users can input the query along with the error and the database they used.
            \item The system should be able to correct typos in column, table, and schema names.
            \item The system should be able to correct incomplete syntax, including issues like missing or extra commas, parentheses, forgotten keywords, misspelled keywords, incorrect order, and missing keywords.
            \item The system should be able to correct incorrect alias formats and add aliases for improved readability.
            \item The system should be able to correct invalid date formats, mismatched data types, and ambiguous column names.
        \end{itemize}

        \item  Convert SQL query to another database
        \begin{itemize}
            \item Users can input the query and the database syntax they are using, as well as the database syntax they want to convert to.
            \item The system should be able to convert syntax from one database to another within our supported databases.
        \end{itemize}

        \item Optimize SQL query
        \begin{itemize}
            \item Users can input the query and optionally with specific aspects they want to optimize.
            \item The system should be able to optimize the query reducing both time and memory consumption.
        \end{itemize}
        \item Translate natural language to SQL query
        \begin{itemize}
            \item Users can input text to specify the data they want to retrieve.
            \item The system should be able to select the appropriate columns and tables and construct the query without errors, ensuring optimal efficiency.
        \end{itemize}
        \item Supported Database
        \begin{itemize}
            \item Impala
            \item StarRocks
            \item Vertica
            \item Spark
        \end{itemize}
    \end{itemize}
    \subsection{Use Cases Diagram and Narrative}
    \begin{figure}[H]
        \centering
        \includegraphics[width=15cm]{chapters/3/figures/usecase_diagram.jpg}
        \caption[Use case diagram of Query Assistance]{Use case diagram of Query Assistance}
        \label{fig:usecase_diagram}
    \end{figure}
    Figure~\ref{fig:usecase_diagram} presents the use case diagram for Query Assistance, visually representing the interactions between the system and its users (actors). It highlights the various functionalities and features available within the Query Assist system, showing how users can engage with it to perform tasks such as generating SQL queries, retrieving data, or integrating with other tools.

    The diagram serves as a high-level overview of the system's capabilities, providing a clear depiction of the relationships between the actors and the system's components. Each use case is further detailed in Tables~\ref{tbl:use-case-fix-sql}-~\ref{tbl:use-case-translate-sql}, where narratives explain the specific functionality, purpose, and flow of each interaction. This comprehensive approach ensures a thorough understanding of how Query Assistance operates and supports user needs. The diagram also emphasizes the flexibility of the system, showcasing its ability to cater to different user roles and scenarios, making it a valuable tool for enhancing productivity and simplifying data-related tasks.

    \begin{table}[H]
        \centering
        \caption{Use case narrative for fix SQL query}
        \label{tbl:use-case-fix-sql}
        \begin{tabular}{|p{4cm}|p{10cm}|}
        \hline
        \textbf{Use Case name} & Fix SQL query \\
        \hline
        \textbf{Actors} & User \\
        \hline
        \textbf{Goal} & Fix SQL Error \\
        \hline
        \textbf{Preconditions} & User has access to the Agoda help platform. \\
        \hline
        \textbf{Main success scenario} &
        \begin{enumerate}
            \item User selects fix query service
            \item User selects query engine
            \item User inputs SQL query and error
            \item System validates the details
            \item System produces the corrected query
        \end{enumerate}
        \\
        \hline
        \textbf{Extension (a)} &
        \begin{enumerate}
            \item[4a.] System validates the details and finds missing required details.
            \item[5a.] System asks the user to fill in the missing required details.
            \item[6a.] Return to step 4
        \end{enumerate}
        \\
        \hline
        \end{tabular}
    \end{table}

    \begin{table}[H]
        \centering
        \caption{Use case narrative for convert SQL query}
        \label{tbl:use-case-convert-sql}
        \begin{tabular}{|p{4cm}|p{10cm}|}
        \hline
        \textbf{Use Case name} & Convert SQL query \\ \hline
        \textbf{Actors} & User \\ \hline
        \textbf{Goal} & Convert SQL to another database syntax \\ \hline
        \textbf{Preconditions} & User has access to the Agoda help platform. \\ \hline
        \textbf{Main success scenario} &
        \begin{enumerate}
            \item User selects convert query service
            \item User selects current query engine
            \item User selects query engine to convert to
            \item System validates the details
            \item System produces the converted query
        \end{enumerate}
        \\ \hline
        \textbf{Extension (a)} &
        \begin{enumerate}
            \item[4a.] System validates the details and finds missing required details.
            \item[5a.] System asks the user to fill in the missing required details.
            \item[6a.] Return to step 4
        \end{enumerate}
        \\ \hline
        \end{tabular}
    \end{table}

    \begin{table}[H]
        \centering
        \caption{Use case narrative for optimize SQL query}
        \label{tbl:use-case-optimize-sql}
        \begin{tabular}{|p{4cm}|p{10cm}|}
        \hline
        \textbf{Use Case name} & Optimize SQL query \\ \hline
        \textbf{Actors} & User \\ \hline
        \textbf{Goal} & Optimize SQL query \\ \hline
        \textbf{Preconditions} & User has access to the Agoda help platform. \\ \hline
        \textbf{Main success scenario} &
        \begin{enumerate}
            \item User selects optimize query service
            \item User selects query engine
            \item User inputs SQL query
            \item System validates the details
            \item System produces the optimized query
        \end{enumerate}
        \\ \hline
        \textbf{Extension (a)} &
        \begin{enumerate}
            \item[4a.] System validates the details and finds missing required details.
            \item[5a.] System asks the user to fill in the missing required details.
            \item[6a.] Return to step 4
        \end{enumerate}
        \\ \hline
        \textbf{Extension (b)} &
        \begin{enumerate}
            \item[3b.] User inputs SQL query and details on optimizing the query
            \item[4b.] Return to step 4
        \end{enumerate}
        \\ \hline
        \end{tabular}
    \end{table}
    \begin{table}[H]
        \centering
        \caption{Use case narrative for translate SQL query}
        \label{tbl:use-case-translate-sql}
        \begin{tabular}{|p{4cm}|p{10cm}|}
        \hline
        \textbf{Use Case name} & Translate SQL query \\ \hline
        \textbf{Actors} & User \\ \hline
        \textbf{Goal} & Translate SQL query \\ \hline
        \textbf{Preconditions} & User has access to the Agoda help platform. \\ \hline
        \textbf{Main success scenario} &
        \begin{enumerate}
            \item User selects translate query service
            \item User selects query engine
            \item User inputs natural language to specify data to retrieve
            \item System validates the details
            \item System produces the SQL query
        \end{enumerate}
        \\ \hline
        \textbf{Extension (a)} &
        \begin{enumerate}
            \item[4a.] System validates the details and finds missing required details.
            \item[5a.] System asks the user to fill in the missing required details.
            \item[6a.] Return to step 4
        \end{enumerate}
        \\ \hline
        \end{tabular}
    \end{table}
\section{System Architecture}
\begin{figure}[H]
    \centering
    \includegraphics[width=15cm]{chapters/3/figures/architecture.png}
    \caption[Query Assistance Architecture Design]{Query Assistance Architecture Design}
    \label{fig:query_assist_architecture_design}
\end{figure}
Figure~\ref{fig:query_assist_architecture_design} illustrates the relationships between the components in our project. Our system employs a PostgreSQL database to store table and column details, sourcing data from OpenMetaData. When requested, the system retrieves necessary information from the PostgreSQL database, processes it according to user requirements, and generates an SQL statement. Before finalizing and returning the SQL statement, it verifies the statement using the database services provided by the user.

\section{Query Assist Bot}
The Query Assist bot is a Slack bot designed to handle the Slack integration of Query Assist. The bot communicates with Query Assist to send inputs, receive outputs, post results to the Slack channel, collect user feedback, and store all relevant information in the messaging table. In the Agoda Slack community, Opsbot serves as a bot to assist users, meaning that users will always interact with Opsbot when seeking help with queries.

    \subsection{System requirements}
    \begin{itemize}
        \item  The system must make API call to Query Assist whenever it was triggered.
        \item  The system must post the result and feedback form, or error message according to the Query Assist result.
        \item  The system must submit the query assist related data and user feedback to Messaging Table.
        \item  The system must monitor the Slack Channel in order to send feedback to messaging table whenever the submit button of the feedback form is clicked.
    \end{itemize}

    \subsection{System Architecture}
    \begin{figure}[H]
        \centering
        \includegraphics[width=15cm]{chapters/3/figures/bot_architecture.png}
        \caption[Query Assist Bot Architecture Design]{Query Assist Bot Architecture Design}
        \label{fig:bot_architecture_design}
    \end{figure}
    Figure~\ref{fig:bot_architecture_design} illustrates the architectural design of the Query Assist bot and the interactions between its components. The Query Assist bot interacts with four components, namely Opsbot, Slack Channel, Query Assist, and Messaging Table, to achieve its functionality.

    The Query Assist bot is a Slack bot designed to handle the Slack integration of Query Assist. The bot communicates with Query Assist to send inputs, receive outputs, post results to the Slack channel, collect user feedback, and store all relevant information in the messaging table. In the Agoda Slack community, Opsbot serves as a bot to assist users, meaning that users will always interact with Opsbot when seeking help with queries.
    \begin{figure}[H]
        \centering
        \includegraphics[width=15cm]{chapters/3/figures/sequence_diagram.png}
        \caption[Sequence Diagram of Query Assist Bot]{Sequence Diagram of Query Assist Bot}
        \label{fig:sequence_diagram}
    \end{figure}
    Figure~\ref{fig:sequence_diagram} shows the interaction between the user, Opsbot, Query Assist Bot, Query Assist, Slack and Messaging Table during the process of Query Assist Slack integration.

    Here is the breakdown of the sequence:
    \begin{itemize}
        \item  Step 1: The user initiates the retrieval process by call Opsbot.
        \item  Step 2: Opsbot return the Query Help form for user.
        \item  Step 3: User input all the required field in the form
        \item  Step 4: Opsbot then call Query Assist Bot and send along the required fields
        \item  Step 5: Query Assist bot then call Query Assist and send along the required fields
        \item  Step 6: Query Assist then return the result and explanation or the error statement
        \item  Step 7: Query Assist Bot post the result and explanation or the error statement to Slack channel
        \item  Alternatively, If no error is return:
        \begin{itemize}
            \item Step 7: Query Assist Bot then send feedback form to Slack channel
            \item Step 8: Slack then send the form to user
            \item Step 9: User send the feedback to Slack
            \item Step 10: Slack send feedback to Query Assist Bot
            \item Step 11: Query Assist Bot send user feedback and query information to Messaging Table
            \item Step 12: Messaging return with storing status
        \end{itemize}
        \item  Alternatively, If error is return:
        \begin{itemize}
            \item Step 7: Query Assist Bot send query information to Messaging Table
            \item Step 8: Messaging return with storing status
        \end{itemize}
    \end{itemize}
\section{Update and retrieve metadata}
The update and retrieve metadata component is responsible for storing and retrieving metadata in vector format within PostgreSQL. This metadata is used to verify the integrity of schema, table, and column names. Storing it in vector format allows for efficient similarity comparisons.
\begin{figure}[H]
    \centering
    \includegraphics[width=15cm]{chapters/3/figures/retrieve_table.png}
    \caption[Sequence Diagram of Query Assist retrieving metadata]{Sequence Diagram of Query Assist retrieving metadata}
    \label{fig:retrieve_table_sequence_diagram}
\end{figure}
Figure~\ref{fig:retrieve_table_sequence_diagram} shows the interaction between the user, Query Assist, OpenMetadata, PostgreSQL, and OpenAI during the process of retrieving database metadata to use for comparing the integrity of schema, table, and column name.

Here is the breakdown of the sequence:
\begin{itemize}
    \item  Step 1: The user initiates the retrieval process by calling API to Query Assist OpenMetadata
    \item  Step 2: Upon receiving the user's request, the Query Assist sends a request to OpenMetdata, requesting table and column details
    \item  Step 3: OpenMetadata retrieves the data and send the table and column details to the Query Assist back to Query Assist
    \item  Step 4: The Query Assist sends a request to PostgreSQL requesting table and column details that it store
    \item  Step 5: PostgreSQL retrieves the data and send the table and column details to the Query Assist back to Query Assist
    \item  Step 6: Query Assist then compare the differences between Openmeta and PostgreSQL to look for the data that is not yet available in PostgreSQL but available in the OpenMetada
    \item  Step 7: Query Assist then sending the list of updated the table and column to OpenAI
    \item  Step 8: OpenAI then embedded and send the vector of updated the table and column back to Query Assist
    \item  Step 9:  Query Assist update the table\_info in the PostgreSQL according to the updated list
    \item  Step 10: Query Assist update the column\_info in the PostgreSQL according to the updated list
\end{itemize}

\section{Tools}
    \subsection{Search Table}
    \begin{figure}[H]
        \centering
        \includegraphics[width=3cm]{chapters/3/figures/search_table.png}
        \caption[Search Table’s flowchart]{Search Table’s flowchart}
        \label{fig:search_table}
    \end{figure}
    Figure~\ref{fig:search_table} illustrates the Search Table Tool workflow. It is a tool designed to efficiently retrieve tables from the database. It provides information such as the schema, table name, description, impression, and distance. The agent interacts with this tool by using the desired table name as input. This name is embedded into a vector that facilitates the retrieval process by identifying tables in the database with the closest resemblance to the provided name. This functionality enables agents to get table information to perform constructing the query.

    Here is the breakdown of the flow:
    \begin{itemize}
        \item  Embedding table\_search\_term for familiarity search
        \item  Construct query with embedded data for similarity search
        \item  Execute constructed query
        \item  Store information in data structure
        \item  Render data in CSV using Jinja2 Template
        \item  Return result in CSV formatted string
    \end{itemize}

    \subsection{Retrieve Table Information}
    \begin{figure}[H]
        \centering
        \includegraphics[width=3cm]{chapters/3/figures/retrieve_table_info.png}
        \caption[Retrieve Table Information's flowchart]{Retrieve Table Information's flowchart}
        \label{fig:retrieve_table_info}
    \end{figure}
    Figure~\ref{fig:retrieve_table_info} illustrates the  Retrieve Table tool workflow which is one of the significant tools used within the Query Assist. Retrieve Table Information focused on retrieving column data of specific tables. This function provided table information and details of columns in the table included column name, data type, and description of column.

    Here is the breakdown of the flow:
    \begin{itemize}
        \item  Embedding column\_search\_term for familiarity search
        \item  Construct query with schema and table name
        \item  Execute constructed query
        \item  Store information in data structure
        \item  Find similarity between each column and embedded vector
        \item  Render data in CSV using Jinja2 Template
        \item  Return result in CSV formatted string
    \end{itemize}

    \subsection{Validate Query}
    \begin{figure}[H]
        \centering
        \includegraphics[width=8cm]{chapters/3/figures/validate.png}
        \caption[Validate Query’s flowchart]{Validate Query’s flowchart}
        \label{fig:validate}
    \end{figure}
    Figure~\ref{fig:validate} illustrates the workflow of the Validate Query tool. Validate query is the tool to check whether the query is executable or not. The tool will use the query given by the agent to execute which added the ‘EXPLAIN’ keyword into the query.

    Here is the breakdown of the flow:
    \begin{itemize}
        \item  Construct a query with an 'EXPLAIN' keyword.
        \item  Execute the query
        \item  Return result of the query if query has no error
        \item  Return error of the query if query has error
    \end{itemize}
\section{Phase 1}
    \subsection{Main Focus}
    In this phase, we focus on establishing the baseline for Query Assist, ensuring the system can detect and fix SQL errors while also handling syntax conversion between different query dialects. This involves implementing error-handling mechanisms, parsing query structures, and integrating rule-based or AI-driven approaches for syntax correction.

    To enhance query understanding and retrieval-augmented generation (RAG), we implement a PostgreSQL database to store metadata about tables and columns. This structured storage allows the system to retrieve relevant schema details, improving its ability to generate accurate query modifications.

    Additionally, we incorporate prompt tuning techniques to optimize interactions with the language model. By refining prompts, we enhance the model's ability to generate precise query transformations and improve overall system performance.

    By the end of this phase, the system should be capable of accurately identifying errors, converting queries to the appropriate format, and leveraging structured database knowledge to enhance SQL assistance.
    \subsection{Implementation}
    The implementation process of this phase began with constructing a baseline using LangChain and integrating the following tools:

    \begin{enumerate}
        \item \textbf{Retrieve Table Info} \\
        This function relies on Retrieval-Augmented Generation (RAG) to extract relevant table metadata. The implementation starts by constructing an SQL statement to fetch comprehensive table information. After retrieving the data, we apply cosine similarity to filter out only the columns that are most relevant to the search term.

        \item \textbf{Search Table} \\
        Similar to the previous tool, this function also utilizes RAG. The implementation begins with generating an SQL statement to fetch a list of tables. We then use cosine similarity to identify and rank tables that closely match the search term.

        \item \textbf{Validate Query} \\
        This tool connects to the database and executes the query using the \texttt{EXPLAIN} keyword. By analyzing the query execution plan, it determines whether the query contains errors, allowing for early detection and correction.
    \end{enumerate}
    \subsection{Slack Integration}
    The integration with Slack is achieved by developing a custom Slack bot designed to act as an intermediary between Slack and Query Assist. This bot facilitates seamless communication by handling the entire interaction process. When a user submits a query through Slack, the bot captures the input and sends it to Query Assist via an API. Once Query Assist processes the input and returns a response, the bot formats the output in a structured and user-friendly manner before posting it back to the appropriate Slack channel.

    In addition to managing the communication flow, the Slack bot is responsible for storing all relevant information in the messaging table. This includes user inputs, Query Assist responses, and any associated metadata, ensuring that all interactions are logged for future reference, analysis, and improvement of the system. By serving as a bridge between Slack and Query Assist, the bot ensures a streamlined and efficient user experience while maintaining a robust record of all query-related activities.
    \begin{figure}[H]
        \centering
        \includegraphics[width=10cm]{chapters/3/figures/Slack-Logo.png}
        \caption[Slack]{Slack}
        \label{fig:slack-logo}
    \end{figure}
\section{Phase 2}
    \subsection{Main Focus}
    In this phase, the focus is on establishing the foundation for Query Assist to enable query generation from Natural Language inputs. This involves implementing methods for data selection and testing procedures to ensure the relevance and accuracy of generated queries.

    A combination of approaches is adopted, including fine-tuning a language model, leveraging workspace-based metadata, and incorporating conversation-driven query generation. These techniques are designed to refine the system's ability to interpret natural language and construct accurate SQL queries.

    To support schema understanding and selection for query construction, test cases are developed to validate the system's behavior across various scenarios. These test cases ensure robust functionality and reliable performance under different query inputs and metadata constraints.

    By the end of this phase, Query Assist is expected to effectively parse user intents, select relevant tables and columns, and generate precise SQL queries aligned with the provided data context.
    \subsection{Test Case Generation}
        \subsubsection{Table-Informed}
        In this method, test cases are generated by providing the table name to GPT, which then creates test cases based on the table's structure and purpose. Our objective is to generate 200 test cases from the top 100 most-used tables. The process involves the following steps:
        \begin{itemize}
            \item \textbf{Table Description Analysis:} Begin by providing GPT with a description of the table to gather details about its structure and columns.
            \item \textbf{Question Generation:} Prompt GPT to create specific, report-driven questions that focus on insights, trends, or summaries.
            \item \textbf{Specification of Tables and Columns:} For each question, clearly specify the table and the columns required to answer it.
            \item \textbf{Validation of Column Names:} Verify that the column names used in the generated questions actually exist in the table. This step ensures that the columns referenced are valid and present, as GPT may occasionally generate non-existent column names.
        \end{itemize}
        Additionally, real-world use cases derived from Opsbot tickets are provided as example questions to guide the generation process.

        To ensure the test cases reflect real-world usage, personas have been created for 11 departments within Agoda, based on internal knowledge. Each persona represents a specific group of users and their unique perspectives. Questions are generated from the perspective of these personas, ensuring they highlight deeper insights and align with actual use cases. This approach allows the test cases to better mimic real-world scenarios and address the key insights typically sought by users of the table.
        \subsubsection{Query-Informed}
        In this method, test cases are generated by filtering real-world user queries based on the most-used tables. The process aims to address the challenges of generating relevant test cases by maximally utilizing existing data. The process involves the following steps:
        \begin{itemize}
          \item \textbf{Table Scope Filtering:} Start by using OpenMetadata to check if a table has a description and ensure that at least 80\text{\%} of the table's columns are described. Only tables meeting both conditions are considered eligible.
          \item \textbf{Ranking by Usage:} Rank the filtered tables based on their usage frequency. Select the top 100 tables with the highest usage as the foundation for the test case generation.
          \item \textbf{Query Extraction:} Identify real user queries that involve these top 100 tables as input for generating detailed and realistic test cases.
          \item \textbf{Question Generation:} Utilize GPT to transform the filtered queries into specific, actionable test cases. The generated questions focus on insights, trends, or summaries derived from real user interactions.
          \item \textbf{Validation of Tables and Columns:} Verify that the tables and columns referenced in the generated questions exist and align with the actual structures of the involved tables. This step ensures accuracy and minimizes the risk of generating outputs that reference non-existent data.
        \end{itemize}
    \subsection{Chatting}
    A decision has been made to transition from a one-time interaction model to a conversational approach. This change enhances user engagement by transforming interactions into ongoing conversations. A user interface has been developed to test the model, enabling seamless conversational interactions. The model integrates its previous responses into new ones, allowing it to retain context and enabling users to continue the conversation without losing the flow.

    The primary objective of this transition is to allow users to provide feedback on GPT-generated answers. This feedback mechanism increases the likelihood of Query Assist delivering more accurate and precise responses over time, improving its overall effectiveness and reliability.
    \subsection{Workspace}
    Through research, the QueryGPT project \cite{QueryGPT} was identified as operating in a manner similar to this work. Their approach to selecting the appropriate table involves categorizing tables into workspaces, a method that has proven effective for their project. Drawing inspiration from this approach, a similar method was implemented by listing tables along with their descriptions and utilizing GPT to categorize them based on the provided descriptions.

    To further enhance table selection and organization, a two-tier filtering system has been developed. This system consists of the following filters:

    \begin{itemize}
        \item Workspace:

        The first filter, called 'Workspace' represents the primary grouping of tables and serves as the main product to be selected. Multiple workspaces can be chosen if the question or use case spans across them.
            \begin{itemize}
                \item Hotel \\
                Focused on tables containing information or data related to accommodations. This includes a wide range of lodging options such as hotels, hostels, vacation rentals, and other property types.
                \item Flight \\
                Dedicated to tables containing information or data related to flights. This includes data on flight bookings, schedules, pricing, financial transactions, promotions, and partnerships. The workspace supports all aspects of the flight booking service, ensuring comprehensive management of air travel-related data.

                \item {Activity} \\
                Designed for tables containing information or data related to activities and local experiences. This includes details about various activities, tours, and experiences that travelers can book to enhance their trips. The data may cover booking information, activity descriptions, pricing, availability, customer feedback, and promotional campaigns.

                \item {Technology and Development Insights} \\
                Provides a comprehensive view of Agoda's technology infrastructure, development processes, and IT operations. It brings together data from GitLab repositories, CI/CD pipelines, team structures, and system monitoring to support analysis of development efficiency, code quality, and operational health.

                \item {Workforce Analytics} \\
                Focused on employee performance, team structures, and workforce optimization.
            \end{itemize}

        \item Category

        The second filter, called "Category," involves selecting the appropriate category within each workspace. Categories provide a more granular classification of tables within a workspace. If a table does not belong to any category, no category is selected.
    \end{itemize}
    This two-tier system ensures a structured and efficient approach to mapping user questions to the appropriate tables, improving the accuracy and relevance of the results.

    \subsection{Fine-tuning}
    The fine-tuning method was introduced as a key development in this project following extensive research and analysis. This decision arose from observing limitations in existing tools, where search terms often failed to yield the desired tables and schema information. To address this issue, a fine-tuning model was developed to enhance the system’s ability to identify and suggest exact tables relevant to the query requirements.

    This approach builds upon the model's ability to learn from structured schema metadata, enabling it to provide more precise and contextually appropriate table selections. The introduction of fine-tuning lays the foundation for improved query generation, offering greater accuracy compared to relying solely on pre-existing search methods.

    By integrating the fine-tuning framework, the project ensures that Query Assist is capable of addressing specific user needs in table selection, thereby enhancing the reliability and effectiveness of SQL query construction.

% THIS IS AN EXAMPLE. ALL SECTIONS BELOW ARE OPTIONAL. PLEASE CONSULT YOU ADVISOR AND DESIGN YOUR OWN SECTION

% \emph{\textthai{หัวข้อต่าง ๆ ในแต่ละบทเป็นเพียงตัวอย่างเท่านั้น หัวข้อที่จะใส่ในแต่ละบทขึ้นอยู่กับโปรเจคของนักศึกษาและอาจารย์ที่ปรึกษา}}

% Explain the design (how you plan to implement your work) of your project. Adjust the section titles below to suit the types of your work. Detailed physical design like circuits and source codes should be placed in the appendix.

% \section{System Architecture}

%     \begin{table}[!h]
%         \centering
%         \caption{test table x1}\label{tbl:symbols}
%         \begin{tabular}{@{}p{0.07\textwidth}|p{0.7\textwidth}p{0.1\textwidth}}\hline
%         \multicolumn{2}{l}{\textbf{SYMBOL}}  & \textbf{UNIT} \\ \hline
%         $\alpha$ & Test variable\hfill & m$^2$ \\
%         $\lambda$ & Interarrival rate\hfill &  jobs/second\\
%         $\mu$ & Service rate\hfill & jobs/second \\ \hline
%         \end{tabular}
%         %\begin{tabular}{c|c} \hline
%         % $\alpha$ & $\beta$ \\ \hline
%         % $\delta$ & $\mu$ \\ \hline
%         %\end{tabular}
%     \end{table}

% \section{System Specifications and Requirements}

% \pagebreak
% \section{Hardware Module 1}
%     \emptyline 2
%     \subsection{Component 1}
%         \emptyline 2
%     \subsection{Logical Circuit Diagram}
%         \emptyline 2
% \pagebreak
% \section{Hardware Module 2}
%     \emptyline 2
%     \subsection{Component 1}
%         \emptyline 2
%     \subsection{Component 2}
%         \emptyline 2
% \pagebreak
% \section{Path Finding Algorithm}

% \pagebreak
% \section{Database Design}

% \pagebreak
% \section{GUI Design}

% \pagebreak

\chapter{ผลการดำเนินงาน}
\thaijustify{
    ในบทที่ 4 จะกล่าวบรรยายและอภิปรายผล... ของโครงการ...
}
\thaijustify{
    ในปีการศึกษาที่ผ่านมา... ตามในแผน... ทางกลุ่มได้ทำ
}
\section{ผลจากการดำเนินการ}

\pagebreak
\section{ผลลัพธ์จากการทดลองเปิดใช้งานซอฟต์แวร์}
   
\pagebreak
\section{การอภิปรายผลจากการดำเนินการ}

\pagebreak
\chapter{Conclusions}

 \begin{figure}[!h]
\caption{This is how you mention when figure come from internet  \href{https://www.google.com} {https://www.google.com}}\label{fig:x1}
\end{figure}

This chapter is optional for proposal and progress reports but 
is required for the final report.

THIS IS AN EXAMPLE. ALL SECTIONS BELOW ARE OPTIONAL. PLEASE CONSULT YOU ADVISOR AND DESIGN YOUR OWN SECTION

\emph{\textthai{หัวข้อต่าง ๆ ในแต่ละบทเป็นเพียงตัวอย่างเท่านั้น หัวข้อที่จะใส่ในแต่ละบทขึ้นอยู่กับโปรเจคของนักศึกษาและอาจารย์ที่ปรึกษา}}

\section{Problems and Solutions}
State your problems and how you fixed them.

\pagebreak
\section{Future Works}
What could be done in the future to make your projects better.

\pagebreak

%%%%%%%%%%%%%%%%%%%%%%%%%%% Bibliography %%%%%%%%%%%%%%%%%%%%%%%%%%%%
%: Comment this in your report to show only references you have
%: cited. Otherwise, all the references below will be shown.
% \nocite{*}

%: Use the bst files for bibtex bibliography style
%: You must have bib files in the referred directories
%: You may go to file .bbl to manually edit the bib items.

%: Improve url breaks to prevent unnecessary big white spaces in some cases
\makeatletter
\g@addto@macro{\UrlBreaks}{\UrlOrds}
\makeatother

%: # Biblatex
%: Add biblatex for managing bibs
%: Uncomment this if you're using Biblatex
% \printbibliography[heading=bibintoc,title={บรรณานุกรม}]

%: # Latex built-in bibliography
%: Uncomment \bibliographystyle and \bibliography if you're using biblatex above
%: References style
%: Examples: \bibliographystyle{bibstyle/kmutt}
\bibliographystyle{bib/kmutt.bst}

%: References files
%: Examples: \bibliography{example/string,example/string2,example/cpe}
\bibliography{bib/cpe.bib}

%%%%%%%%%%%%%%%%%%%%%%%%%%% Appendices %%%%%%%%%%%%%%%%%%%%%%%%%%%%%%
%: Insert your appendices here the same way you insert content
%: chapters using \input{path\to\appendix\tex}

%: Define new 'appendix' name if there are multiple 'appendices'
% \def\appendixnames{Appendices}

% \appendix{แผนผังและแบบแปลนโครงสร้างซอฟต์แวร์ปรับใหม่}

    \begin{figure}[H]
        \centering
            \includegraphics[width=15cm]{appendices/A/figures/design/navigation-v3.png}
        \label{fig:appendix-nav-map}
        \caption[แผนผังการสัญจรและนำทางหน้าเว็บไซต์ที่มีแบบใหม่]{แผนผังการสัญจรและนำทางหน้าเว็บไซต์ที่มีแบบใหม่}
    \end{figure}
    
    \centering\noindent{\large\bf แผนผังการสัญจรหน้าเว็บไซต์ที่มีการเปลี่ยนแปลงไปจากเดิม แบบใหม่มีการแยก View ของ Organizer กับ User ธรรมดาอย่างชัดเจน ไม่มีการ Switch ไปมาอีกต่อไป} \\

\pagebreak
% \appendix{ตัวอย่างซอร์สโค้ดและชุดคำสั่งในซอฟต์แวร์โครงการ}
\noindent{\large\bf ตัวอย่างการทำ Dependency Injection ระหว่างไฟล์ User's Usecase กับไฟล์ User's Repository} \\

    \begin{lstlisting}[label={lst:appendix-user-domain}, caption={ชุดคำสั่งที่นิยาม Interface ของ User Repository และ User Usecase}]
    // ...
    type UserRepository interface {
        Create(user *User) error
        Get(id string) (*User, error)
        GetBySessionId(id string) (*User, error)
        GetByEmail(email string, provider AuthProvider) (*User, error)
        Update(user *User) error
    }
    
    type UserUsecase interface {
        Create(email string, password string) (*User, error)
        CreateFromGoogle(id string, email string, name string) (*User, error)
        Get(id string) (*User, error)
        GetBySessionId(id string) (*User, error)
        GetByEmail(email string, provider AuthProvider) (*User, error)
        Update(id string, user *UpdateUser) error
        UpdatePassword(id string, oldPlainPassword string, newPlainPassword string) error
    }
    // ...
    \end{lstlisting}
    
    \begin{lstlisting}[label={lst:appendix-user-usecase}, caption={ชุดคำสั่งสร้างและนิยาม UserUsecase ที่มีการ Implement UserRepository}]
    // ...
    type userUsecase struct {
        seaweedfs      *platform.SeaweedFs
        userRepository domain.UserRepository
        sessionUsecase domain.SessionUsecase
    }
    
    func NewUserUsecase(
        seaweedfs *platform.SeaweedFs,
        userRepository domain.UserRepository,
        sessionUsecase domain.SessionUsecase,
    ) domain.UserUsecase {
        return &userUsecase{
            seaweedfs:      seaweedfs,
            userRepository: userRepository,
            sessionUsecase: sessionUsecase,
        }
    }
    // ...
    \end{lstlisting}
    
\pagebreak

\end{document}
%%%%%%%%%%%%%%%%%%%%%%%%%% REPORT END %%%%%%%%%%%%%%%%%%%%%%%%%%%%%%%
