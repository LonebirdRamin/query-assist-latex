\chapter{Introduction}


% \emph{}

\section{Background} 

Undeniably, data plays a critical role in driving companies especially for swiftly moving business worlds. The data could direct the company by offering insights that can solve problems, lead to improvements, help making an informed decision making, understand the customer preference, enhance marketing strategies, cost reduction, risk management, and so many other things. Relying on specialists to retrieve data efficiently is costly and time-consuming. It would therefore be rather advantageous and efficient to create a query support tool that assists users in retrieving data from the database, particularly for fast-growing travel technology companies like Agoda.
The potential benefits of Query Assistance are that it would help in simplifying data access by enabling employees to quickly retrieve data without requiring an understanding of SQL. The user can retrieve data quickly and efficiently using the Query Assistance in which helps to save cost, time, and resources from querying the data.

\section{Proposed Method}
    \subsection{Approach}
    Query Assistance is the tool for creating the SQL query based on the user input. The Query Assistance is capable of debugging  the SQL query, optimizing  the SQL query, and translating the natural language to the SQL query. Initially, the Query Assistance will be built as an API platform, allowing other platforms to call it for execution. It will then be integrated into Superset, enabling users to utilize it when they encounter issues in SQL Lab. The Query Assistance works by using Large Language Models (LLMs) and LangChain as a core function. Large Language Models (LLMs) play a crucial role in this project by improving and expediting SQL operations. The LLMs will be specifically used to fix errors in SQL code, ranging from syntax problems to incorrect table or column references, and automatically generate SQL queries based on natural language input. They will also be used to modify existing SQL statements to improve accuracy or optimize performance Complementing the capabilities of LLMs, LangChain provides a robust framework for building applications where these models can function as an agent. The agent performs tasks such as understanding user inputs, generating responses, and interacting with data. LangChain also offers a library for creating custom tools, aiming to optimize the use of agents and tools to achieve project objectives. The agents extend their capabilities beyond mere text generation by interacting with tools that enable them to fetch real-time data, perform computations, and execute specific commands. To further enhance data interaction, LangChain utilizes metadata stored in OpenMetadata to analyze and select the appropriate columns and tables. This integration ensures that the agents can make informed decisions and effectively manage data interactions, thereby achieving the project's goals efficiently.
    \subsection{Objectives}
        \begin{itemize}
        \item  Simplify data access: Empower workers to retrieve data easily without needing to understand SQL.
        \item  Enhance query accuracy: Automatically correct SQL statements, from syntax to selecting the correct columns and tables.
        \item  Boost efficiency: Save time by quickly generating flawless SQL queries, optimizing both productivity and cost.
        \item  Facilitate advanced queries: Allow users to effortlessly create complex queries and explore advanced SQL techniques with minimal errors.
        \end{itemize}

\section{Scope of Work}
This project will be divided into three phases. The first phase focuses on error handling to ensure that we ultimately provide the correct query for the user. This includes correcting all syntax errors, accommodating syntax differences across databases, and fixing column names, table names, and typos in the query. The second phase focuses on optimizing the query such as using the partition column, decreasing memory consumption, and reducing the query execution time. The third and final phase aims to translate natural language into SQL query language.

\section{Original Engineering Content}
The original engineering content covered in the project includs the following: Software Engineering, Data Analysis, AI, and User Interface 
    \begin{itemize}
        \item  Software
        \begin{itemize}
            \item API Design and Implementation
        \end{itemize}
        \item  Data Analysis
        \begin{itemize}
            \item Performance Metrics
        \end{itemize}
        \item  System Integration
        \begin{itemize}
            \item Superset Integration
            \item Slack Integration
        \end{itemize}
    \end{itemize}

\pagebreak
\section{Project Schedule}
    \begin{table}[!h]
        \centering % Ensures the table is centered
        \caption{Project schedule in the first semester}
        \label{tbl:gantt1}
        \resizebox{\textwidth}{!}{ % Resizes the Gantt chart to fit within the page width
            \begin{ganttchart}[
                x unit = 0.5cm,
                y unit chart = 1.2cm,
                y unit title = 0.6cm,
                title height = 1,
                vgrid={*{3}{black, dotted}, *1{black, dashed}},
                hgrid={*1{black, dashed}},
                bar top shift = 0.1, 
                bar height = 0.8,
                bar label font=\small, % Adjust font size for labels
                bar label node/.append style={
                    align=right,
                    text width=6cm % Set a fixed width for text wrapping
                }
            ]{1}{20}
                \gantttitle{Aug}{4} \gantttitle{Sep}{4} \gantttitle{Oct}{4} \gantttitle{Nov}{4} \gantttitle{Dec}{4} \\
                \gantttitlelist{1,...,4}{1} \gantttitlelist{1,...,4}{1} \gantttitlelist{1,...,4}{1} \gantttitlelist{1,...,4}{1} \gantttitlelist{1,...,4}{1} \\
                \ganttbar{Project Discussion with company}{1}{1} \\
                \ganttbar{Project Idea}{2}{2} \\
                \ganttbar{Project Brainstorm}{2}{3} \\
                \ganttbar{Design architecture for project}{3}{4} \\
                \ganttbar{Set up project}{4}{4} \\
                \ganttbar{Design sequence diagram for create API to add and update Table details}{5}{6} \\
                \ganttbar{Implement query assist table details API}{4}{4} \\
                \ganttbar{Design sequence diagram for tools in Query Assist}{5}{6} \\
                \ganttbar{Implement tools for Query Assist}{7}{7} \\
                \ganttbar{Merge and Debug}{7}{8} \\
                \ganttbar{Prompt Tuning \& Monitor result}{8}{9} \\
                \ganttbar{CI/CD + Deploy}{9}{9} \\
                \ganttbar{Slack Integration}{10}{15} \\
                \ganttbar{Slack Tracking Performance + Enhance the MVP}{12}{13} \\
                \ganttbar{Merge PostgreSQL}{14}{15} \\
                \ganttbar{Merge Query Assist v1 with v2}{16}{17} \\
                \ganttbar{POC and Design the Optimization function}{15}{15} \\
                \ganttbar{Implement Optimization function}{18}{20}  % No extra \\
            \end{ganttchart}
        }
    \end{table}
\pagebreak
    \begin{table}[!h]
        \centering
        \caption{Project schedule in the second semester}
        \label{tbl:gantt2}
        \begin{ganttchart}[
            x unit = 0.5cm,
            y unit chart = 1.2cm,
            y unit title = 0.6cm,
            title height = 1,
            vgrid={*{3}{black, dotted}, *1{black, dashed}},
            hgrid={*1{black, dashed}},
            bar top shift = 0.1, 
            %bar height = 0.8,
            bar label node/.append style={
                align=right,
                % text width=width("Present findings to advisors")
            }
        ]{1}{20}
            \gantttitle{Jan}{4} \gantttitle{Feb}{4} \gantttitle{Mar}{4} \gantttitle{Apr}{4} \gantttitle{May}{4} \\
            \gantttitlelist{1,...,4}{1} \gantttitlelist{1,...,4}{1} \gantttitlelist{1,...,4}{1} \gantttitlelist{1,...,4}{1} \gantttitlelist{1,...,4}{1} \\
            \ganttbar{Implement Optimization function}{1}{2} \\
            \ganttbar{Prompt Tuning \& Monitor result}{3}{5} \\
            \ganttbar{Superset Integration}{4}{5} \\
            \ganttbar{Get feedback + Enhance the functions}{5}{5} \\
            \ganttbar{POC and Design the Phase 3 functions}{6}{7} \\
            \ganttbar{Implement Phase 3 functions}{8}{14} \\
            \ganttbar{Prompt Tuning \& Monitor result}{14}{15} \\
            \ganttbar{Get feedback + Enhance the functions}{15}{17} \\
            \ganttbar{Implement Multi Agent}{16}{18} \\
            \ganttbar{Debug}{18}{19} \\
            \ganttbar{Improve and Finalize}{19}{20}  % No extra \\
        \end{ganttchart}
    \end{table}
\section{Deliverables for Term 1}
\begin{itemize}
    \item  Overall system design
    \begin{itemize}
        \item Architecture Design
        \item Use Case Diagram
        \item Sequence Diagram
        \item Flow Chart
    \end{itemize}

    \item  Query Assist Phase 1
    \begin{itemize}
        \item Correcting all syntax errors
        \item Accommodating syntax differences across databases
        \item Fixing typos in column names, and table names
    \end{itemize}

    \item  Integrate into Slack
    \item Deploy MVC (Minimum viable product)
\end{itemize}
\section{Deliverables for Term 2}
\begin{itemize}
    \item  Performance Metrics

    \item  Query Assist Phase 2
    \begin{itemize}
        \item Auto add partition column
        \item Decreasing memory consumption
        \item Reducing the query execution time
    \end{itemize}
    \item Query Assist Phase 3
    \begin{itemize}
        \item Translate natural language to SQL query
    \end{itemize}
\end{itemize}
\pagebreak