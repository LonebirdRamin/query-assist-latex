\chapter{Conclusions}

% \begin{figure}[!h]
% \caption{This is how you mention when figure come from internet  \href{https://www.google.com} {https://www.google.com}}\label{fig:x1}
% \end{figure}

\section{Introduction}
In this chapter, we will conclude about the project outcomes in terms of our work progress. We also discuss the problems we encountered during the project and the solutions we used to overcome those problems. The last part is our opinion about how this project can be improved in the future.
\section{Problems and Solutions}
\subsection{Project Initiation Phase} At the outset, there was uncertainty regarding the appropriate steps to initiate the project, leading to hesitation and lack of direction.

\underline{\textbf{Solution:}} Project plans were discussed with the assigned mentor, and a comprehensive diagram was developed to ensure alignment and clarity among all stakeholders.

\subsection{Communication Enhancement Phase} Communication barriers were identified, resulting in misunderstandings and reduced team efficiency.

\underline{\textbf{Solution:}} Active efforts were made to improve communication channels, fostering better understanding and minimizing the risk of miscommunication.

\subsection{Knowledge Acquisition Phase} Gaps in knowledge within specific technical areas hindered progress and limited the ability to make informed decisions.

\underline{\textbf{Solution:}} Targeted research and practical test runs were conducted to deepen understanding of the system and address knowledge deficiencies.

\subsection{Terminology Familiarization Phase} Unfamiliarity with domain-specific terminology, particularly terms used by Agoda, created confusion and slowed progress.

\underline{\textbf{Solution:}} Continuous learning and proactive inquiry were employed to clarify unfamiliar terms, resulting in increased familiarity with the relevant processes.

\subsection{Time Management} Excessive time was being allocated to certain tasks, negatively impacting overall project timelines.

\underline{\textbf{Solution:}} Timetables and time management strategies were refined to enhance productivity and ensure more efficient allocation of resources.



\section{Work Progress}
\begin{table}[H]
    \centering
    \caption{Action Items}
    \label{tbl:task-breakdown-status}
    \begin{tabular}{|p{6cm}|p{3cm}|p{5cm}|}
        \hline
        \textbf{Action Items} & \textbf{Progress} & \textbf{Substituted Approaches}  \\
        \hline
        Project Discussion & Complete &  \\
        \hline
        Idea Generation & Complete &  \\
        \hline
        Brainstorming & Complete & \\
        \hline
        System Design & Complete & \\
        \hline
        Project Setup & Complete & \\
        \hline
        Table API Sequence Diagram & Complete & \\
        \hline
        Build Table API & Complete & \\
        \hline
        Query Tool Diagram & Complete & \\
        \hline
        Build Query Tools & Complete & \\
        \hline
        Merge \& Debug & Complete & \\
        \hline
        Tune \& Monitor  & Complete & \\
        \hline
        CI/CD \& Deploy & Complete & \\
        \hline
        Slack Integration & Complete & \\
        \hline
        Slack Tracking \& MVP & Complete & \\
        \hline
        Merge Database V1/V2 & Complete & \\
        \hline
        Merge Code V1/V2 & Complete & \\
        \hline
        Phase 2 Design/POC & Complete & \\
        \hline
        Build Optimizer & Complete & \\
        \hline
        POC Generating Feature & Complete & \\
        \hline
        Evaluate \& Merge Models & Complete & \\
        \hline
        Improve Phase 1 & Complete. & \\
        \hline
        Superset + Deploy MVC & Not Complete & Agoda GPT integration\\
        \hline
        Feedback \& Enhance & Complete &  \\
        \hline
        POC \& Design Optimization feature & Complete & \\
        \hline
        Implement Optimization & Not complete & Implement Data Question function\\
        \hline
        Phase 2 Prompt Tuning \& Monitor & Not complete & Implement Data Question function\\
        \hline
        Phase 2 Feedback \& Enhance & Not complete & Implement Data Question function\\
        \hline
        Finalizing and completing the project & Complete & \\
        \hline
    \end{tabular}
\end{table}
\pagebreak
\section{Future Works}
\begin{itemize}
  \item Enhancing Metadata for Improved Table Selection:\\The current results indicate that the metadata stored in our system is insufficient for Query Assist to accurately select the most appropriate tables when generating SQL queries. Metadata such as comprehensive table descriptions, detailed column annotations, data lineage, and usage examples are often lacking or too generic. By systematically improving and enriching the metadata—particularly through more informative table and column descriptions, as well as explicit tagging of business context—we can significantly enhance Query Assist's ability to understand the purpose and content of each table. This, in turn, would lead to more relevant table selection and overall improved query generation accuracy for end users.
  \item Augmenting Query Assist with Business and Performance Knowledge:\\Another limitation observed is that Query Assist is currently unable to address query performance issues or recommend optimal strategies for complex queries. This is largely due to its limited awareness of table relationships, data volume characteristics, and advanced SQL optimization techniques specific to our business context. Enabling Query Assist with richer business knowledge—such as common usage patterns, best practices for joins and indexing, and awareness of performance-critical tables—could empower it to provide actionable recommendations for query optimization. Furthermore, incorporating feedback mechanisms or learning from query execution results could allow Query Assist to continuously improve its optimization strategies over time. Ultimately, expanding Query Assist’s knowledge base in these areas can provide users with more powerful tools for both generating and refining their queries, leading to more efficient and effective data analysis.
\end{itemize}
\pagebreak
